\documentclass[a4paper, 11pt]{report}
\usepackage{blindtext}
\usepackage[T1]{fontenc}
\usepackage[utf8]{inputenc}
\usepackage{titlesec}
\usepackage{fancyhdr}
\usepackage{geometry}
\usepackage{fix-cm}
\usepackage[hidelinks]{hyperref}
\usepackage{graphicx}
\usepackage{titlesec}

\usepackage[english]{babel}

\geometry{ margin=30mm }
\counterwithin{subsection}{section}
\renewcommand\thesection{\arabic{section}.}
\renewcommand\thesubsection{\thesection\arabic{subsection}.}
\usepackage{tocloft}
\renewcommand{\cftchapleader}{\cftdotfill{\cftdotsep}}
\renewcommand{\cftsecleader}{\cftdotfill{\cftdotsep}}
\setlength{\cftsecindent}{2.2em}
\setlength{\cftsubsecindent}{4.2em}
\setlength{\cftsecnumwidth}{2em}
\setlength{\cftsubsecnumwidth}{2.5em}

\titlespacing\section{0pt}{12pt plus 4pt minus 2pt}{0pt plus 2pt minus 2pt}
\titlespacing\subsection{0pt}{12pt plus 4pt minus 2pt}{0pt plus 2pt minus 2pt}

\begin{document}
\titleformat{\section}
{\normalfont\fontsize{15}{0}\bfseries}{\thesection}{1em}{}
\titlespacing{\section}{0cm}{0.5cm}{0.15cm}
\titleformat{\subsection}
{\normalfont\fontsize{13}{0}\bfseries}{\thesubsection}{0.5em}{}
\titlespacing{\section}{0cm}{0.5cm}{0.15cm}

%=============================================================================

\pagenumbering{Alph}
\begin{titlepage}
\begin{flushright}
\includegraphics[width=4cm]{USyd}\\[2cm]
\end{flushright}
\center 
\textbf{\huge INFO1111: Computing 1A Professionalism}\\[0.75cm]
\textbf{\huge 2023 Semester 1}\\[2cm]
\textbf{\huge Self-Learning Report}\\[3cm]

\textbf{\huge Submission number: 2}\\[0.75cm]
\textbf{Github link: https://github.com/abra8541/Info1111Submission.git}\\[2cm]

{\large
\begin{tabular}{|p{0.35\textwidth}|p{0.55\textwidth}|}
	\hline
	{\bf Student name} & Ameek Brar\\
	{\bf Student ID} & 530471181\\
	{\bf Topic} & Javascript\\
	{\bf Levels already achieved} & Level A\\
	{\bf Levels in this report} & Level B and C\\
	\hline
\end{tabular}
}
\thispagestyle{empty}
\end{titlepage}
\pagenumbering{arabic}


%=============================================================================

\tableofcontents

%=============================================================================

\newpage
\section*{Instructions}

\textbf{Important}: This section should be removed prior to submission.

You should use this \LaTeX\ template to generate your self-learning report. Keep in mind the following key points:
\begin{itemize}
	\item \textbf{Submissions}: There will be three opportunities during the semester to submit this report. For each submission you can attempt 1 or 2 levels. Each submission should use the same report, but amended to include new information.
	\item \textbf{Assessment}: In order to achieve level B, you must first have achieved level A, and so on for each level up to level D. This means that we will not assess a higher level until a lower level has been achieved (though we will review one level higher and give you feedback to help you in refining your work).
	\item \textbf{Minimum requirement}: Remember that in order to pass the unit, you must achieve at least level A in self-learning (unless you achieve level B in both the skills and knowledge categories).
	\item \textbf{Using this template}: When completing each section you should remove the explanation text and replace it with your material.
	\item \textbf{Referencing}: You should also ensure that any resources you use are suitably referenced, and references are included in the reference list at the end of this document. You should use the IEEE reference style \cite{usyd2} (the reference included here shows you how this can be easily achieved).
\end{itemize}


%=============================================================================


\newpage
\section{Level A: Initial Understanding}
\vspace{5mm}
\subsection{Level A Demonstration}
1. Understand how JavaScript works with HTML to produce a web page myself and embed a JS file  \\ 2. Understand and recreate basic functions such as alert, console.log, and document.write \\ 3. Create a multi-layer interactive calculator using a base of JavaScript and implement it into a web page using HTML and CSS.

\subsection{Learning Approach}
Being a beginner in JavaScript, I began my journey in understanding the new language by reading basic summary articles as to what JavaScript is and its role in technological development.  This helped build my base knowledge as to "Why JavaScript is important?." From that point, I watched YouTube tutorials on understanding key parts of JavaScript such as outputting code into an HTML web page, as well as developing the correct syntactical method of writing the code. Since the process of learning to code is project-based, I tried to recreate code from YouTube tutorials myself in a JavaScript file. (In my case it was my Hello World page). This process allowed me to develop my ability to make simple lines of code correctly following required syntax and function rules. Continuing on I watched more tutorials and free online programs from learning platforms to challenge myself further and gain in-depth knowledge. Other approaches to learning involved speaking to family relatives who have learnt the language and asking them about the best methods to pick up skills in JavaScript. This helped me find websites that have clear details on Functions such as console.log, alert and document. write and where they are expressed on a real-life web page. Moving forward I will be using more online resources such as websites and even textbooks that demonstrate more complex ideas such as operators which will help me create my own project later on.   

\subsection{Challenges and Difficulties}
Whilst learning aspects of JavaScript there were many resources available to me such as YouTube tutorials, videos, academic guides, textbooks, etc., I did struggle at one point to identify what sources were the best to learn from since even though some resources were described as for beginners, they involved in-depth experience of coding in general maybe from another language such as Python, which I do not have. This challenge was often seen by websites that stated learning python beforehand I beneficial for JavaScript. However, I was able to overcome this issue by looking for more replacement resources and looking for keywords such as "New to Coding" and "Beginner in Coding", which helped target me by the ability of skills I had. Another difficulty I'm struggling with is how there are multiple ways to achieve the same outcome in coding. When inputting JavaScript into an HTML file there were 2 methods I came across in my research, such as importing the actual JS file or actually writing the JS code directly into the HTML file. I was able to rectify this issue by referring back to my demonstration component of Level A where I clarified that JavaScript is my main focus. As a result, the method I have used to code my JavaScript is by creating it in a separate file that I can later import.

\subsection{Learning Sources}
Sources used to develop my Knowledge. Includes multi-forms of resources I used to expand my understanding such as websites and videos.

\begin{tabular}{|p{0.47\textwidth}|p{0.45\textwidth}|}
	\hline
	https://www.w3schools.com/js/ & Developed my understanding of basic levels and steps of JavaScript by working down the contents page to understand different concepts such as inputs and outputs using correct syntax  \\
	\hline
	https://www.simplilearn.com/applications-of-javascript-article & Helped clarify worldwide uses of JS and its applications in the real-world \\
	\hline
	https://developer.mozilla.org/en-US/docs/Web/API/Window/alert & Developed my interpretation on the alert() function in JavaScript and how it can be used to allow interaction with a user in a webpage\\
	\hline
	https://developer.mozilla.org/en-US/docs/Web/API/Console/log/ & Aided my research in understanding the function of Console.log() in JS and how it can be viewed in an HTML file via inspect -> console\\
	\hline
	https://www.youtube.com/@cemeygimedia & Helped me understand the contribution of HTML in representing a JS code file, and how it can be used to create webpages in an efficient manner.\\
	\hline
\end{tabular}

\cite{LevelA1}
\cite{LevelA2}
\cite{LevelA3}
\cite{LevelA4}
\cite{LevelA5}

\subsection{Application artifacts}
I re-created my own version of a YouTube tutorial that I watched to understand the basic concepts of JavaScript and how it works in conjunction with HTML. I created an HTML template and a JS code file which I later imported to create an active web page which I viewed using the VS code extension of Live Server, where I demonstrated the functions of the console.log() to see its presence in the web page console. I also created an alert() to see how it interacts with a user once executed. Additionally, I completed my first code of "Hello World" using the document.write() function in JS, making it visible to a user in the web page.   


%=============================================================================

\newpage
\section{Level B: Basic Application}

Whilst level A is about doing something simple with the topic to just show that you have started to be able to use the tool or technology, level B is about doing something practical that might actually be useful.

\subsection{Level B Demonstration}

In Level B I have created a simple basic functioning calculator that can be used to solve basic math problems involving functions such as addition, multiplication, subtraction and division. This helps demonstrate my understanding of utilising JavaScript as the main code and embedding it with a CSS file and an HTML file to create a calculator website. 

\subsection{Application artifacts}

To complete Level B I created a simple Calculator, it solves simple mathematical equations involving addition, subtraction, multiplication and division. It works by the user pressing the desired keys using their cursor to input and output a result. I created this using a main HTML file that I then embedded with a JavaScript file that then was able to run the main functions of the calculator, however, to present the calculator in a neat way I used a CSS file, that I had to learn to use in conjunction with the other 2 files to allow the calculator to be used on the HTML file. 


%=============================================================================

\newpage
\section{Level C: Deeper Understanding}

Level C focuses on showing that you have actually understood the tool or technology at a relatively advanced level. You will need to compare it to alternatives, identifying key strengths and weaknesses, and the areas where this tool is most effective. 

\subsection{Strengths}
The strength of the simple functioning calculator I created using JavaScript, HTML and CSS, is that it is simple to understand and actually utilise the project. For a user who needs to access the web page to work out simple calculations, it can be done very quickly and efficiently without them being confused which is a huge factor that must be fulfilled with coding projects as many times they can be hard to execute. Additionally, for me, it was quite versatile to create as I could design things such as the colours and sizing very easily through my CSS file after watching a few tutorials on YouTube.

\subsection{Weaknesses}
With my project, a weakness I noticed was that the calculator is currently minimal, it doesn't have very advanced levels of mathematical computational skills that a regular Casio calculator has. In order to fix this I could add trigonometric functions as well as square root buttons and other functions seen in a physical calculator. With the actual creation process, JavaScript has a weakness I identified in the early stage of building. JavaScript is a dynamically typed language, which means that variable types can change at run time, making it more difficult to catch errors during development. In order to overcome this I had to double-check my code whilst typing it to prevent as many errors as possible before running it.

\subsection{Usefulness}
The topic of using an HTML that embedded JavaScript and CSS files is very important as it allows programmers to create actual webpages that are interactive for a user. This can increase the aesthetic appeal of simple games and functioning codes in Javascript which is an essential factor in the Software development industry when publicizing a new project to the public. The calculator itself is also very useful as it helps solve math problems any individual may have in everyday needs such as homework, assignments or even grocery shopping. 
\cite{LevelC1}

\subsection{Key Question 1: When should you use javascript, or not use javascript?}
A variety of web development scenarios can use JavaScript, which is a strong language. However, there are some circumstances in which using JavaScript might not be required or appropriate. You can use JavaScript when creating interactive user experiences. It is ideal for creating interactive and dynamic user experiences on the web, such as animations, form validation, and other client-side functionality. But security is a big issue for JavaScript, if your website handles sensitive information such as credit card details or personal information, you may need to be cautious, as JavaScript can be vulnerable to security attacks such as cross-site scripting.
\cite{LevelC2}

\subsection{Key Question 2: How does JavaScript make use of frameworks?}
JavaScript utilises frameworks as already written code storages that provide developers with a set of tools, functions, and structures to help them build complex applications more efficiently with the time and lines of code needed. JavaScript frameworks can be used for both front-end and back-end building and can help simplify the development process by providing ready-made solutions for common programming tasks. Frameworks like React, Angular, and Vue can also help improve code organization. By leveraging the features of JavaScript frameworks, developers can save time and effort in building applications, as well as making sure their code is correct and follows best practices.
\cite{LevelC3}

%=============================================================================

\newpage
\section{Level D: Evolution of skills}
\vspace{5mm}
\subsection{Level D Demonstration}

This is a short description of the application that you have developed. (50-100 words).
\textit{{\bf IMPORTANT:} You might wish to submit this as part of an earlier submission in order to obtain feedback as to whether this is likely to be acceptable for level D.}

\subsection{Application artifacts}

Include here a description of what you actually created (what does it do? How does it work? How did you create it?). Include any code or other related artefacts that you created (these should also be included in your github repository).

If you do include screengrabs to show what you have done then these should be annotated to explain what it is showing and what the application does.

\subsection{Alternative tools/technologies}
Identify 2 alternative tools/technologies that can be used instead of the one you studied for your topic. (e.g. if your topic was Python, then you might identify Java and Golang)
\subsection{Comparative Analysis}
Describe situations in which both your topic and each of the identified alternatives would be preferred over the others (100-200 words).



%=============================================================================

\newpage

\bibliographystyle{IEEEtran}
\bibliography{main}

\end{document}
\end{report}
